%\documentclass[portrait,final,a0paper,fontscale=0.292]{baposter}
\documentclass[portrait,a0paper,fontscale=0.31]{baposter}

\usepackage{calc}
\usepackage{graphicx}
\usepackage{amsmath}
\usepackage{amssymb}
\usepackage{relsize}
\usepackage{multirow}
\usepackage{rotating}
\usepackage{bm}
\usepackage{url}

% Redmar
\usepackage{enumitem}       %set indent for enumerate/itemize
\setitemize{leftmargin=0.3cm,itemsep=1.3em,parsep=0pt}%,topsep=0pt}
\usepackage[none]{hyphenat} %disables all hyphenation
%https://tex.stackexchange.com/questions/79591/superscripts-in-bibliography-with-bibtex
\usepackage[super]{natbib} 
\usepackage[font=small,labelfont=bf]{caption}
\DeclareCaptionFont{tiny}{\tiny}
%\usepackage{svg}
\usepackage{verbatim}

\usepackage{graphicx}
\usepackage{multicol}

%\usepackage{times}
%\usepackage{helvet}
%\usepackage{bookman}
\usepackage{palatino}

% Disable due to conflict with usepackage caption
%\newcommand{\captionfont}{\footnotesize}

\graphicspath{{images/}{../images/}}
\usetikzlibrary{calc}

%% Some names etc
\newcommand{\senterica}{\textit{Salmonella enterica} serovar Heidelberg }
\newcommand{\heidelberg}{\textit{S.} Heidelberg }
\newcommand{\fcite}[2]{#1$^#2$}
\newcommand{\mdr}{multidrug-resistant }
\newcommand{\Mdr}{Multidrug-resistant }
\newcommand{\reference}{reference strain SL476 }
\newcommand{\amr}[2]{\textbf{#1} #2,}
\newcommand{\famr}[2]{\textbf{#1} #2.}
\newcommand{\cmy}{\textit{bla}{\tiny CMY-2}}
\newcommand{\padd}{\vspace*{0.2cm}}
\newcommand{\ppadd}{\vspace*{0.29cm}}

%%%%%%%%%%%%%%%%%%%%%%%%%%%%%%%%%%%%%%%%%%%%%%%%%%%%%%%%%%%%%%%%%%%%%%%%%%%%%%
%%% Begin of Document
%%%%%%%%%%%%%%%%%%%%%%%%%%%%%%%%%%%%%%%%%%%%%%%%%%%%%%%%%%%%%%%%%%%%%%%%%%%%%%

\begin{document}

%%%%%%%%%%%%%%%%%%%%%%%%%%%%%%%%%%%%%%%%%%%%%%%%%%%%%%%%%%%%%%%%%%%%%%%%%%%%%%
%%% Here starts the poster
%%%---------------------------------------------------------------------------
%%% Format it to your taste with the options
%%%%%%%%%%%%%%%%%%%%%%%%%%%%%%%%%%%%%%%%%%%%%%%%%%%%%%%%%%%%%%%%%%%%%%%%%%%%%%
% Define some colors

%\definecolor{lightblue}{cmyk}{0.83,0.24,0,0.12}
\definecolor{lightblue}{rgb}{0.145,0.6666,1}
\definecolor{darkblue}{rgb}{.082352,.2588235,.450980}

%\hyphenation{resolution occlusions}
%%
\begin{poster}%
  % Poster Options
  {
  % Show grid to help with alignment
  grid=false,
  % Column spacing
  colspacing=2em,
  % Color style
  bgColorOne=white,
  bgColorTwo=white,
  %borderColor=lightblue,
  borderColor=white,
  headerColorOne=darkblue,
  headerColorTwo=darkblue,
  headerFontColor=white,
  boxColorOne=white,
  boxColorTwo=darkblue,
  % Format of textbox
  textborder=faded,
  % Format of text header
  eyecatcher=true,
  headerborder=closed,
  headerheight=0.1\textheight,
%  textfont=\sc, An example of changing the text font
  headershape=rounded,
  headershade=shadelr,
  headerborder=open,
  headerfont=\Large\sf\bf, %Sans Serif
  headerheight=0.13\textheight,
  textfont={\setlength{\parindent}{1.5em}},
  boxshade=plain,
%  background=shade-tb,
  background=plain,
  linewidth=2pt,
  }
  % Eye Catcher
  %{\includegraphics[height=5em]{images/nvwa.png}} 
  {}
  % Title
  {
    {
      ExonViz: An application to visualize transcripts and variants
    }
    %\vspace{0.3em}
  }
  % Authors
  {
    %\vspace{0.15em}
    {
      \fcite{Redmar R. van den Berg}{{1,2,3}}, \fcite{Marlen C. Lauffer}{{1,2}}, \fcite{Jeroen F.J. Laros}{{2,4}}
      \\{\smaller $^1$ \textit{Dutch Center for RNA Therapeutics}}
      \\{\smaller $^2$ \textit{Department of Human Genetics, Leiden University Medical Center, Leiden, The Netherlands}}
      \\{\smaller $^3$ \textit{Department of Hematology, Leiden University Medical Center, Leiden, The Netherlands}}
      \\{\smaller $^4$ \textit{Department of Bioinformatics and Computational Services, National Institute of Public Health and the Environment (RIVM), The Netherlands}}
    }
    %\vspace{0.15em}
  }
  %{Netherlands Food and Consumer Product Safety Authority}
  % University logo
  %{\includegraphics[height=10.5em]{images/rijksoverheid3.png}} 
  %{\includegraphics[height=9.5em]{images/rijksoverheid3.png}} 
  % {\parbox[top][12em][t]{5em}{\includegraphics[height=9.5em]{images/rijksoverheid4.png}}} 

%%%%%%%%%%%%%%%%%%%%%%%%%%%%%%%%%%%%%%%%%%%%%%%%%%%%%%%%%%%%%%%%%%%%%%%%%%%%%%
%%% Now define the boxes that make up the poster
%%%---------------------------------------------------------------------------
%%% Each box has a name and can be placed absolutely or relatively.
%%% The only inconvenience is that you can only specify a relative position 
%%% towards an already declared box. So if you have a box attached to the 
%%% bottom, one to the top and a third one which should be in between, you 
%%% have to specify the top and bottom boxes before you specify the middle 
%%% box.
%%%%%%%%%%%%%%%%%%%%%%%%%%%%%%%%%%%%%%%%%%%%%%%%%%%%%%%%%%%%%%%%%%%%%%%%%%%%%%
    % % A coloured circle useful as a bullet with an adjustably strong filling
    %\newcommand{\colouredcircle}{%
    %  \tikz{\useasboundingbox (-0.2em,-0.32em) rectangle(0.2em,0.32em); \draw[draw=black,fill=lightblue,line width=0.03em] (0,0) circle(0.18em);}}

%%%%%%%%%%%%%%%%%%%%%%%%%%%%%%%%%%%%%%%%%%%%%%%%%%%%%%%%%%%%%%%%%%%%%%%%%%%%%%
  \headerbox{Introduction}{name=introduction,column=0,row=0}{
%%%%%%%%%%%%%%%%%%%%%%%%%%%%%%%%%%%%%%%%%%%%%%%%%%%%%%%%%%%%%%%%%%%%%%%%%%%%%%
    \ppadd
    \noindent Don't draw exons by hand, use our tool
 }


%%%%%%%%%%%%%%%%%%%%%%%%%%%%%%%%%%%%%%%%%%%%%%%%%%%%%%%%%%%%%%%%%%%%%%%%%%%%%%
  \headerbox{Features}{name=features,below=introduction} {
    \padd
    \begin{itemize}
      \item Draw transcripts including features like exon frames, coding region and variants
      \item All exon sizes and variant positions are biologically accurate
      \item Draw transcripts for any species
      \item If a gene name is specified that exists in human, the MANE Select transcript will be used by default.
      \item Extensive documentation with examples
      \item Specify custom / novel / patient transcripts using a simple Excel file.
    \end{itemize}
  }
  \headerbox{Materials and Methods}{name=MM,below=features} {
    \padd
    \begin{itemize}
      \item ExonViz is written in Flask
      \item It uses the Mutalyzer (CITE) API to fetch transcript information
      \item It has the ability to split large exons so they fit on the specified page
    \end{itemize}
  }

%%%%%%%%%%%%%%%%%%%%%%%%%%%%%%%%%%%%%%%%%%%%%%%%%%%%%%%%%%%%%%%%%%%%%%%%%%%%%%
%%%%%%%%%%%%%%%%%%%%%%%%%%%%%%%%%%%%%%%%%%%%%%%%%%%%%%%%%%%%%%%%%%%%%%%%%%%%%%
  \headerbox{Examples}{name=examples,column=1,span=2} {
  % \noindent{\centering\includegraphics[width=0.75\linewidth]{images/Tree8_1.png}\\}
  \captionsetup{labelformat=empty,margin=0.6cm}
  \noindent{\centering\includegraphics[width=0.75\linewidth]{images/SDHD.png}\\}
  \captionof{figure}{Transcript NM\_003002.4 for SDHD, with two variants}
  }

    %\caption{SMX: Sulfamethoxazole; CIP: Ciprofloxacin; NAL: Nalidixic Acid; TET: Tetracycline; AMP: Ampicillin; CHL: Chloramphenicol; FOT: Cefotaxime; TAZ: Ceftazidime; GEN: Gentamicin; COL: Colistin}

%%%%%%%%%%%%%%%%%%%%%%%%%%%%%%%%%%%%%%%%%%%%%%%%%%%%%%%%%%%%%%%%%%%%%%%%%%%%%%
  \headerbox{References}{name=ref,column=2,above=bottom} {
%%%%%%%%%%%%%%%%%%%%%%%%%%%%%%%%%%%%%%%%%%%%%%%%%%%%%%%%%%%%%%%%%%%%%%%%%%%%%%
    \tiny
    \begin{enumerate}[itemsep=-0.3ex,leftmargin=0.29cm]
      \item Lefter, M., et al. "Mutalyzer 2: next generation HGVS nomenclature checker, Bioinformatics (2021). https://doi.org/10.1093/bioinformatics/btab051
    \end{enumerate}
  }

%%%%%%%%%%%%%%%%%%%%%%%%%%%%%%%%%%%%%%%%%%%%%%%%%%%%%%%%%%%%%%%%%%%%%%%%%%%%%%
\headerbox{Contact}{name=contact,column=1, aligned=ref}
%%%%%%%%%%%%%%%%%%%%%%%%%%%%%%%%%%%%%%%%%%%%%%%%%%%%%%%%%%%%%%%%%%%%%%%%%%%%%%
  {
  \begin{center}
  \vspace{1.25em}

  \textbf{RedmarvandenBerg@lumc.nl}\linebreak
  \textbf{DCRT@lumc.nl}\linebreak
%\linebreak\linebreak\linebreak\linebreak
  \end{center}
  }


%%%%%%%%%%%%%%%%%%%%%%%%%%%%%%%%%%%%%%%%%%%%%%%%%%%%%%%%%%%%%%%%%%%%%%%%%%%%%%
\headerbox{QR Code}{name=qr,column=0,aligned=ref}
%%%%%%%%%%%%%%%%%%%%%%%%%%%%%%%%%%%%%%%%%%%%%%%%%%%%%%%%%%%%%%%%%%%%%%%%%%%%%%
  {
  %hfill to align right: https://tex.stackexchange.com/questions/55472/how-to-make-text-aligned-left-center-right-in-the-same-line
  \vspace{-0.3em}
  \begin{center}
  %\includegraphics[width=0.25\linewidth]{images/meegid_poster2016.png}
  % \includegraphics[width=0.25\linewidth]{images/meegid_heidelberg_poultry.png}
  \end{center}
  }
\end{poster}

\end{document}

